
\documentclass[a4paper]{article}\usepackage[]{graphicx}\usepackage[]{color}
%% maxwidth is the original width if it is less than linewidth
%% otherwise use linewidth (to make sure the graphics do not exceed the margin)
\makeatletter
\def\maxwidth{ %
  \ifdim\Gin@nat@width>\linewidth
    \linewidth
  \else
    \Gin@nat@width
  \fi
}
\makeatother

\definecolor{fgcolor}{rgb}{0.345, 0.345, 0.345}
\newcommand{\hlnum}[1]{\textcolor[rgb]{0.686,0.059,0.569}{#1}}%
\newcommand{\hlstr}[1]{\textcolor[rgb]{0.192,0.494,0.8}{#1}}%
\newcommand{\hlcom}[1]{\textcolor[rgb]{0.678,0.584,0.686}{\textit{#1}}}%
\newcommand{\hlopt}[1]{\textcolor[rgb]{0,0,0}{#1}}%
\newcommand{\hlstd}[1]{\textcolor[rgb]{0.345,0.345,0.345}{#1}}%
\newcommand{\hlkwa}[1]{\textcolor[rgb]{0.161,0.373,0.58}{\textbf{#1}}}%
\newcommand{\hlkwb}[1]{\textcolor[rgb]{0.69,0.353,0.396}{#1}}%
\newcommand{\hlkwc}[1]{\textcolor[rgb]{0.333,0.667,0.333}{#1}}%
\newcommand{\hlkwd}[1]{\textcolor[rgb]{0.737,0.353,0.396}{\textbf{#1}}}%

\usepackage{framed}
\makeatletter
\newenvironment{kframe}{%
 \def\at@end@of@kframe{}%
 \ifinner\ifhmode%
  \def\at@end@of@kframe{\end{minipage}}%
  \begin{minipage}{\columnwidth}%
 \fi\fi%
 \def\FrameCommand##1{\hskip\@totalleftmargin \hskip-\fboxsep
 \colorbox{shadecolor}{##1}\hskip-\fboxsep
     % There is no \\@totalrightmargin, so:
     \hskip-\linewidth \hskip-\@totalleftmargin \hskip\columnwidth}%
 \MakeFramed {\advance\hsize-\width
   \@totalleftmargin\z@ \linewidth\hsize
   \@setminipage}}%
 {\par\unskip\endMakeFramed%
 \at@end@of@kframe}
\makeatother

\definecolor{shadecolor}{rgb}{.97, .97, .97}
\definecolor{messagecolor}{rgb}{0, 0, 0}
\definecolor{warningcolor}{rgb}{1, 0, 1}
\definecolor{errorcolor}{rgb}{1, 0, 0}
\newenvironment{knitrout}{}{} % an empty environment to be redefined in TeX

\usepackage{alltt}

\usepackage{hyperref}
\usepackage{graphicx}
\usepackage{longtable}
\usepackage{underscore}
\usepackage[utf8]{inputenc}
\usepackage[spanish]{babel}

\newcommand{\etal}{\emph{et alt.}}
\newcommand{\Rpackage}[1]{{\texttt{#1}}}
\newcommand{\Rcode}[1]{{\texttt{#1}}}
\newcommand{\Rfun}[1]{{\texttt{#1}}}

\newcommand{\R}{{\it R}}


\title{An\'alisis estad\'istico para la selecci\'on \\ 
       de miRNAs diferencialmente expresados \\ 
       tras un infarto cerebral, en diferentes regiones \\
       del cerebro. UEB ID: A-XXXX}

\author{Ricardo Gonzalo y Alex Sánchez}

\bibliographystyle{plain}
\IfFileExists{upquote.sty}{\usepackage{upquote}}{}
\begin{document}
\setkeys{Gin}{width=0.7\textwidth} % Sets default width for R-Sweave generated figures

\maketitle
\newpage
\tableofcontents
\newpage
\section{Introducci\'on}

\subsection{Objetivos}

El principal objetivo de este estudio es encontrar miRNAs diferencialmente expresados en muestras procedentes de tres regiones de necropsias de cerebro, en pacientes que habían fallecido debido a un infarto cerebral o ictus. Las tres zonas del cerebro que se estudian son las siguientes:
\begin {itemize}
  \item Contralateral (CL): zona del cerebro sana, alejada del lugar del infarto.
  \item Periinfarto (PI): zona de alrededor del infarto.
  \item Infarto (IC): zona donde ha ocurrido el infarto.
\end {itemize}

\subsubsection{Condiciones Experimentales}
Las condiciones experimentales a considerar en este estudio son las siguientes:
\begin{enumerate}
  \item Zona de estudio:
    \begin{itemize}
      \item Periinfarto (PI)
      \item Contralateral (CL)
      \item Infarto (IC)
    \end{itemize}
  \item Conservación del tejido:
    \begin{itemize}
      \item Tejido fresco (FF)
      \item Tejido conservado en parafina (FFPE)
      \item Tejido conservado en paraformaldehido (PFA)
    \end{itemize}
\end{enumerate}

\subsection{An\'alisis}

El an\'alisis se ha realizado siguiendo un \textbf{``pipeline''} más o
menos estándar de an\'alisis de datos de microarrays, adaptado a este
estudio en particular.  En l\'ineas generales cada apartado de los
siguientes se corresponde con una secci\'on de este informe.
\begin{enumerate}
\item Obtención de los datos.
\item Control de Calidad: Se valora la calidad individual y global de
  los arrays. 
\item Preprocesado de los datos: En general consiste en la
  sumarizaci\'on, normalizaci\'on y filtraje. En este caso los datos
  se han proporcionado ya normalizados, por lo que tan sólo se añade
  el filtraje.
\item Selecci\'on de genes diferencialmente expresados para cada una de las condiciones descritas.
\item B\'usqueda de patrones de expresi\'on comunes y agrupaci\'on de las muestras.
\item Anotaci\'on de los resultados y visualización de patrones de expresión.
\end{enumerate}


En \textit{Bioinformatics and Computational Biology Solutions using
  {R} and Bioconductor, Gentleman R et al, Springer (2005)} (\cite{Gentleman:2005}) se pueden
encontrar descritos los m\'etodos utilizados para procesar, analizar e
interpretar los datos. Para realizar este estudio se ha utilizado el
software {R} y los paquetes apropiados del Proyecto Bioconductor
(\textit{www.bioconductor.org})

\subsection{Archivos y presentación de resultados}

El estudio realizado genera una considerable cantidad de
archivos. Para facilitar su revisión se proporciona un archivo
comprimido que, al extraerlo, contiene un archivo
\texttt{Results.html} (que puede abrirse con un navegador como CHROME
o EXPLORER) y que da acceso a los distintos archivos generados en el
proceso del an\'alisis agrupados por categorías (información, datos,
control de calidad, análisis, anotación).

Las tablas de resultados se proporcionan en formato html y en formato
CSV (``comma separated values'') que pueden abrirse con una hoja de
cálculo indicando que los separadores son comas (``,'').










\section{Obtención y lectura de los datos}

Para un análisis de datos de microarrays de Affymetrix se necesitan los archivos 
de imágenes escaneadas (``.CEL'') y un archivo en el que se asigne una condición 
experimental a cada archivo (archivo ``targets'').

\subsection{Los datos}

Los archivos .CEL han sido procesados con el software ``Expression
Console'' de Affymetrix que ha permitido leerlos y normalizarlos
utilizando el algoritmo RMA (\cite{Irizarry2003}) el habitual para este tipo de arrays.

Los datos normalizados se encuentran almacenados en el archivo
\texttt{rma.summary.txt}.

\subsection{Lectura de los datos}

La lectura de datos se lleva a cabo utilizando las clases y métodos
definidas en los paquetes \Rcode{Biobase} y \Rcode{affy} de
Bioconductor.

Una forma cómoda de leer los datos y, al mismo tiempo, asignar a cada
muestra los valores de las covariables (por ejemplo el grupo para el
analisis) consiste en crear un pequeño archivo de texto, que suele
denominarse \Rcode{targets.csv}. EL archivo targets para este caso
tiene el aspecto siguiente:

La tabla \ref{table.targets} muestra el contenido del archivo \texttt{targets.csv}
con la asignación de cada muestra a cada condición experimental.

{
 \tiny
% latex table generated in R 3.2.0 by xtable 1.7-4 package
% Thu Jul  9 16:18:09 2015
\begin{longtable}{rlllllllrlr}
  \hline
 & Patient & Gender & Area & Method & Grupo & ShortName & ColoresArea & pchArea & ColoresMethod & pchMethod \\ 
  \hline
16\_C\_1.CEL & 16 & F & CL & FF & CL.FF & P16.F.CL.FF & green &   1 & green &   1 \\ 
  16\_C\_2.CEL & 16 & F & CL & PFA & CL.PFA & P16.F.CL.PFA & green &   1 & red &   2 \\ 
  16\_C\_3.CEL & 16 & F & CL & FFPE & CL.FFPE & P16.F.CL.FFPE & green &   1 & blue &   3 \\ 
  16\_I\_1.CEL & 16 & F & IC & FF & IC.FF & P16.F.IC.FF & red3 &   2 & green &   1 \\ 
  16\_I\_2.CEL & 16 & F & IC & PFA & IC.PFA & P16.F.IC.PFA & red3 &   2 & red &   2 \\ 
  16\_I\_3.CEL & 16 & F & IC & FFPE & IC.FFPE & P16.F.IC.FFPE & red3 &   2 & blue &   3 \\ 
  19\_C\_1.CEL & 19 & F & CL & FF & CL.FF & P19.F.CL.FF & green &   1 & green &   1 \\ 
  19\_C\_2.CEL & 19 & F & CL & PFA & CL.PFA & P19.F.CL.PFA & green &   1 & red &   2 \\ 
  19\_C\_3.CEL & 19 & F & CL & FFPE & CL.FFPE & P19.F.CL.FFPE & green &   1 & blue &   3 \\ 
  19\_I\_1.CEL & 19 & F & IC & FF & IC.FF & P19.F.IC.FF & red3 &   2 & green &   1 \\ 
  19\_I\_2.CEL & 19 & F & IC & PFA & IC.PFA & P19.F.IC.PFA & red3 &   2 & red &   2 \\ 
  19\_I\_3.CEL & 19 & F & IC & FFPE & IC.FFPE & P19.F.IC.FFPE & red3 &   2 & blue &   3 \\ 
  22\_C\_1.CEL & 22 & M & CL & FF & CL.FF & P22.M.CL.FF & green &   1 & green &   1 \\ 
  22\_C\_2.CEL & 22 & M & CL & PFA & CL.PFA & P22.M.CL.PFA & green &   1 & red &   2 \\ 
  22\_C\_3.CEL & 22 & M & CL & FFPE & CL.FFPE & P22.M.CL.FFPE & green &   1 & blue &   3 \\ 
  22\_I\_1.CEL & 22 & M & IC & FF & IC.FF & P22.M.IC.FF & red3 &   2 & green &   1 \\ 
  22\_I\_2.CEL & 22 & M & IC & PFA & IC.PFA & P22.M.IC.PFA & red3 &   2 & red &   2 \\ 
  22\_I\_3.CEL & 22 & M & IC & FFPE & IC.FFPE & P22.M.IC.FFPE & red3 &   2 & blue &   3 \\ 
  22\_P\_1.CEL & 22 & M & PI & FF & PI.FF & P22.M.PI.FF & blue &   3 & green &   1 \\ 
  22\_P\_2.CEL & 22 & M & PI & PFA & PI.PFA & P22.M.PI.PFA & blue &   3 & red &   2 \\ 
  22\_P\_3.CEL & 22 & M & PI & FFPE & PI.FFPE & P22.M.PI.FFPE & blue &   3 & blue &   3 \\ 
  32\_C\_1.CEL & 32 & M & CL & FF & CL.FF & P32.M.CL.FF & green &   1 & green &   1 \\ 
  32\_C\_2.CEL & 32 & M & CL & PFA & CL.PFA & P32.M.CL.PFA & green &   1 & red &   2 \\ 
  32\_C\_3.CEL & 32 & M & CL & FFPE & CL.FFPE & P32.M.CL.FFPE & green &   1 & blue &   3 \\ 
  32\_I\_1.CEL & 32 & M & IC & FF & IC.FF & P32.M.IC.FF & red3 &   2 & green &   1 \\ 
  32\_I\_2.CEL & 32 & M & IC & PFA & IC.PFA & P32.M.IC.PFA & red3 &   2 & red &   2 \\ 
  32\_I\_3.CEL & 32 & M & IC & FFPE & IC.FFPE & P32.M.IC.FFPE & red3 &   2 & blue &   3 \\ 
  33\_C\_1.CEL & 33 & M & CL & FF & CL.FF & P33.M.CL.FF & green &   1 & green &   1 \\ 
  33\_C\_2.CEL & 33 & M & CL & PFA & CL.PFA & P33.M.CL.PFA & green &   1 & red &   2 \\ 
  33\_C\_3.CEL & 33 & M & CL & FFPE & CL.FFPE & P33.M.CL.FFPE & green &   1 & blue &   3 \\ 
  33\_I\_1.CEL & 33 & M & IC & FF & IC.FF & P33.M.IC.FF & red3 &   2 & green &   1 \\ 
  33\_I\_2.CEL & 33 & M & IC & PFA & IC.PFA & P33.M.IC.PFA & red3 &   2 & red &   2 \\ 
  33\_I\_3.CEL & 33 & M & IC & FFPE & IC.FFPE & P33.M.IC.FFPE & red3 &   2 & blue &   3 \\ 
  33\_P\_1.CEL & 33 & M & PI & FF & PI.FF & P33.M.PI.FF & blue &   3 & green &   1 \\ 
  33\_P\_2.CEL & 33 & M & PI & PFA & PI.PFA & P33.M.PI.PFA & blue &   3 & red &   2 \\ 
  33\_P\_3.CEL & 33 & M & PI & FFPE & PI.FFPE & P33.M.PI.FFPE & blue &   3 & blue &   3 \\ 
  35\_C\_1.CEL & 35 & M & CL & FF & CL.FF & P35.M.CL.FF & green &   1 & green &   1 \\ 
  35\_C\_2.CEL & 35 & M & CL & PFA & CL.PFA & P35.M.CL.PFA & green &   1 & red &   2 \\ 
  35\_C\_3.CEL & 35 & M & CL & FFPE & CL.FFPE & P35.M.CL.FFPE & green &   1 & blue &   3 \\ 
  35\_I\_1.CEL & 35 & M & IC & FF & IC.FF & P35.M.IC.FF & red3 &   2 & green &   1 \\ 
  35\_I\_2.CEL & 35 & M & IC & PFA & IC.PFA & P35.M.IC.PFA & red3 &   2 & red &   2 \\ 
  35\_I\_3.CEL & 35 & M & IC & FFPE & IC.FFPE & P35.M.IC.FFPE & red3 &   2 & blue &   3 \\ 
  35\_P\_1.CEL & 35 & M & PI & FF & PI.FF & P35.M.PI.FF & blue &   3 & green &   1 \\ 
  35\_P\_2.CEL & 35 & M & PI & PFA & PI.PFA & P35.M.PI.PFA & blue &   3 & red &   2 \\ 
  35\_P\_3.CEL & 35 & M & PI & FFPE & PI.FFPE & P35.M.PI.FFPE & blue &   3 & blue &   3 \\ 
  36\_C\_1.CEL & 36 & F & CL & FF & CL.FF & P36.F.CL.FF & green &   1 & green &   1 \\ 
  36\_C\_2.CEL & 36 & F & CL & PFA & CL.PFA & P36.F.CL.PFA & green &   1 & red &   2 \\ 
  36\_I\_1.CEL & 36 & F & IC & FF & IC.FF & P36.F.IC.FF & red3 &   2 & green &   1 \\ 
  36\_I\_2.CEL & 36 & F & IC & PFA & IC.PFA & P36.F.IC.PFA & red3 &   2 & red &   2 \\ 
  C1\_CON\_3.CEL & C1AP & F & CON & FFPE & CON.FFPE & PC1AP.F.CON.FFPE & black &   4 & blue &   3 \\ 
  C2\_CON\_3.CEL & C2AP & F & CON & FFPE & CON.FFPE & PC2AP.F.CON.FFPE & black &   4 & blue &   3 \\ 
  C3\_CON\_1.CEL & C3 & F & CON & FF & CON.FF & PC3.F.CON.FF & black &   4 & green &   1 \\ 
  C3\_CON\_2.CEL & C3 & F & CON & PFA & CON.PFA & PC3.F.CON.PFA & black &   4 & red &   2 \\ 
  C4\_CON\_1.CEL & C4 & - & CON & FF & CON.FF & PC4.-.CON.FF & black &   4 & green &   1 \\ 
  C4\_CON\_2.CEL & C4 & - & CON & PFA & CON.PFA & PC4.-.CON.PFA & black &   4 & red &   2 \\ 
   \hline
\hline
\caption{Archivo de asignación a cada muestra de su condición experimental} 
\label{table.targets}
\end{longtable}

}
\normalsize




\section{Exploración, Control de Calidad y Normalización}

Tras leer los datos pasamos al preprocesado.
Aunque puede interpretarse de distintas formas esta fase suele consistir en 
\begin{enumerate}
\item Realizar algunos gráficos con los datos para hacerse una idea de como ha resultado el experimento.
\item Realizar un control de calidad más formal.
\item Normalizar y, en el caso de Affymetrix, resumir las expresiones
\end{enumerate}

\subsection{Exploración y visualización}

La exploración previa puede hacerse paso a paso o en bloque si se utiliza algún paquete como
\Rpackage{arrayQualityMetrics}. En la ayuda de este paquete se encuentra una descripción de los análisis básicos que permite realizar.

El diagrama de cajas (\ref{fig:boxPlot}), como el histograma, da una idea de la distribución de los datos.

\begin{knitrout}
\definecolor{shadecolor}{rgb}{0.969, 0.969, 0.969}\color{fgcolor}\begin{figure}
\includegraphics[width=\maxwidth]{images/graficboxPlot-1} \caption[Intensity distribution of normalized data]{Intensity distribution of normalized data}\label{fig:boxPlot}
\end{figure}


\end{knitrout}
\noindent Como se puede observar en el gráfico la distribución de la intensidad entre las muestras es muy diferente, parece ser principalmente debido al método de conservación de las muestras (Tejido fresco = FF, Paraformaldehido= PFA, Parafinado= FFPE). Esto lo observaremos mejor en el cluster jerárquico y en el análisis de componentes principales.





Un cluster jerárquico seguido de un dendrograma nos puede ayudar a hacernos una idea
de si las muestras se agrupan por condiciones experimentales (\ref{fig:plotDendro}). 

Si lo hacen es bueno, pero si no, no es necesariamente indicador de problemas, 
puesto que es un gráfico basado en todos los datos.

\begin{knitrout}
\definecolor{shadecolor}{rgb}{0.969, 0.969, 0.969}\color{fgcolor}\begin{figure}
\includegraphics[width=\maxwidth]{images/graficplotDendro-1} \caption[Hierarchical clustering of samples]{Hierarchical clustering of samples}\label{fig:plotDendro}
\end{figure}


\end{knitrout}





El gráfico anterior muestra una agrupación en tres grandes bloques: en el primero de ellos, y muy diferente de los otros dos, se agrupan las muestras conservadas en FFPE. Después hay un grupo donde parece que se agrupan las muestras de infarto (IC), mientras que en el otro gupo parecen agruparse las muestras procedentes de la zona contralateral (CL).

Otra forma de observar la agrupación de las muestras es realizar un Análisis de Componentes Principales (PCA). Se han realizado dos PCA, uno con los colores de las muestras definidos según el área del cerebro de la biopsia (\ref{fig:plotPCA2DArea}), y el otro según el método de conservación (\ref{fig:plotPCA2DMethod}),  de manera que se facilite la visualización de la agrupación de las muestras. Se puede observar en el PCA según el método de conservación (\ref{fig:plotPCA2DMethod}) que la principal fuente de variabilidad (primera componente, 46.2\% de variabilidad explicada)  es el método de conservación de la biopsia, agrupando las FFPE a un lado y el resto (PFA y FF) al otro lado. El segundo componente parece explicar el área dónde se ha extraído la biopsia (\ref{fig:plotPCA2DArea}), ya que parece haber agrupar las muestras de infarto (IC) a un lado y las otras al otro lado. Es importante comentar que el método de conservación en parafina (FFPE), enmascara totalmente el valor de la variable \texttt{Area}.



\begin{knitrout}
\definecolor{shadecolor}{rgb}{0.969, 0.969, 0.969}\color{fgcolor}\begin{figure}
\includegraphics[width=\maxwidth]{images/graficplotPCA2DArea-1} \caption[PCA plot of samples]{PCA plot of samples: 2 first principal components.}\label{fig:plotPCA2DArea}
\end{figure}


\end{knitrout}



\begin{knitrout}
\definecolor{shadecolor}{rgb}{0.969, 0.969, 0.969}\color{fgcolor}\begin{figure}
\includegraphics[width=\maxwidth]{images/graficplotPCA2DMethod-1} \caption[PCA plot of samples]{PCA plot of samples: 2 first principal components.}\label{fig:plotPCA2DMethod}
\end{figure}


\end{knitrout}





\subsection{Control de calidad}

El control de calidad consiste en una serie de gráficos que buscan sobretodo detectar variaciones excepcionales en los datos (``outliers'') suceptibles de ser eliminados del análisis.

El paquete \Rpackage{arrayQualityMetrics} encapsula los análisis que
pueden realizarse con distintos paquetes de forma que con una
instrucción se pueden realizar todos los análisis y enviar la salida a
un archivo.






Como resultado de la ejecución se crea una carpeta \texttt{QualityControl} que contiene un archivo \texttt{index.html} que da acceso a todos gráficos de control de calidad. A este archivo se accede tambien desde el navegador de resultados. La interpretación de cada gráfico se lleva a cabo mediante la información disponible al pie de las figuras, pero, a modo de resumen puede afirmarse \emph{que los datos disponibles son de buena calidad y sin presencia de valores extremos}.

\subsubsection{PVCA}
En este apartado se ha usado el paquete \textit{PVCA} de \textit{Bioconductor}, el cual combina dos m\'etodos com\'unmente usados, el an\'alisis de componentes principales (PCA) y el an\'alisis de componentes de la varianza. Con esto podemos determinar cuales son las fuentes de variaci\'on que m\'as afectan, cuantificando la proporcion de variabilidad explicada por cada variable.

\begin{knitrout}
\definecolor{shadecolor}{rgb}{0.969, 0.969, 0.969}\color{fgcolor}\begin{figure}
\includegraphics[width=\maxwidth]{images/graficplotPVCA-1} \caption[PVCA plot of variable weight]{PVCA plot of variable weight.}\label{fig:plotPVCA}
\end{figure}


\end{knitrout}

Como se observa en la figura \ref{fig:plotPVCA} la variable \textit{Method} (forma de conservación de las biopsias) es la que explica más variabilidad de todas las que se han utilizado en el modelo. De todas maneras sigue quedando una gran cantidad (62\%)de variabilidad sin explicar.

Para concluir con esta seccion merece la pena tener en cuenta que todos estos graficos son exploratorios. Raramente se descarta un array si solo un grafico lo sugiere.

\subsection{Normalizacion y Filtraje}

Una vez realizado el control de calidad se puede realizarse un
filtraje no específico con el fin de eliminar genes que constituyen
básicamente ``ruído'', bien porque sus señales son muy bajas o bien
porque apenas varían entre condiciones, por lo que no aportan nada a
la selección de genes diferencialmente expresados.

\subsubsection{Filtraje}

El filtraje \emph{no específico} permite eliminar los genes que varían poco entre condiciones o que deseamos quitar por otras razones como por ejemplo que no disponemos de anotación para ellos. Este proceso permite eliminar los genes que, o bien varían poco, o bien no se dispone de anotación para ellos.

%La función \texttt{nsFilter} devuelve los valores filtrados en un objeto \texttt{expressionSet} y un informe de los resultados del filtraje. 


En total en este estudio se han eliminado \textbf{3419} transcritos por considera que no variaban mínimamente entre condiciones lo que implica que no muestren expresión diferencial. Se continua trabajando con \textbf{3212} genes.



\section{Selección de genes diferencialmente expresados}

Como en las etapas anteriores la selección de genes diferencialmente
expresados (GDE) puede basarse en distintas aproximaciones, desde la
$t$ de Student al programa SAM pasando por multitud de variantes.

En este ejemplo se aplicará la aproximación presentada por Smyth
\emph{et al.} (2004) basado en la utilización del \emph{modelo lineal
  general} combinada con un método para obtener una estimación
mejorada de la varianza.


\subsection{Análisis basado en modelos lineales}

La selección de genes diferecialmente expresados se ha basado en la
utilización de métodos similares al análisis de la varianza (ANOVA)
desarrollados explícitamente para el análisis de microarrays.

Dichos métodos, implementados en el paquete \Rpackage {limma} de
Bioconductor (\url{http://bioinf.wehi.edu.au/limma/},
\cite{Smyth:2004}), amplían el análisis tradicional utilizando modelos
de Bayes empíricos para combinar la información de toda la matriz de
datos y de cada gen individual y obtener estimaciones de error
mejoradas.

\begin{knitrout}
\definecolor{shadecolor}{rgb}{0.969, 0.969, 0.969}\color{fgcolor}\begin{kframe}


{\ttfamily\noindent\itshape\color{messagecolor}{\#\# Loading required package: limma\\\#\# \\\#\# Attaching package: 'limma'\\\#\# \\\#\# The following object is masked from 'package:BiocGenerics':\\\#\# \\\#\#\ \ \ \  plotMA}}\end{kframe}
\end{knitrout}

\subsubsection{Comparaciones realizadas}

Los análisis estadísticos de microarrays suelen basarse en realizar
comparaciones o \emph{contrastes} entre condiciones
experimentales. Cada comparación genera una lista de genes que puede
estudiarse posteriormente de diversas formas.

Las comparaciones pueden agruparse en \emph{grupos de comparaciones}
que, no son más que comparaciones relacionadas, por ejemplo las
distintas dosis del mismo fármaco comparadas con el control. Dadas
varias comparaciones, que generan varias listas de genes se pueden
relacionar de diversas formas:
\begin{enumerate}
\item Podemos buscar en que difieren ambas listas, lo cual se
  realizará mediante un test estadístico (y una lista adicional),
\item Podemos buscar en que se parecen, es decir que genes aparecen en común en ambas listas. En este caso no realizaremos ningún test y tan sólo compararemos las listas. 
\end{enumerate}

En este caso interesa estudiar:
\begin{itemize}
\item Diferencias entre IC y CL para cada método de conservación.
  \begin{enumerate}
    \item IC.FF vs CL.FF = IC.FF - CL.FF
    \item IC.FFPE vs CL.FFPE = IC.FFPE - CL.FFPE
    \item IC.PFA vs CL.PFA = IC.PFA - CL.PFA 
  \end{enumerate}
\item Diferencias entre métodos de conservación en las muestras CL
   \begin{enumerate}
    \item CL.FFPE vs CL.FF = CL.FFPE - CL.FF 
    \item CL.PFA vs CL.FF =  CL.PFA - CL.FF
    \item CL.PFA vs CL.FFPE = CL.PFA - CL.FFPE 
  \end{enumerate}
\item Diferencias entre métodos de conservación en las muestras IC
\begin{enumerate}
    \item IC.FFPE vs IC.FF = IC.FFPE - IC.FF 
    \item IC.PFA vs IC.FF =  IC.PFA - IC.FF
    \item IC.PFA vs IC.FFPE = IC.PFA - IC.FFPE 
  \end{enumerate}
\end{itemize}



\subsubsection{Estimación del modelo y selección de genes}

Una vez definida el modelo de
análisis % (la matriz de diseño y los contrastes)
podemos pasar a %estimar el modelo, estimar los contrastes y
realizar las pruebas de significación que nos indiquen, para cada gen
y cada comparación, si puede considerarse diferencialmente expresado.




El análisis proporciona los estadísticos de test habituales con
ciertas variaciones que son propias del tipo de test utilizado:
\begin{itemize}
\item valores de $t$ (t-moderado calculado por el paquete \texttt{limma} para ser exacto)
\item diferencia de expresión media, tambien llamado \texttt{Fold--change} o \texttt{logFC} ya que se calcula en escala logarítmica.
\item $p$-valores, sin ajustar y ajustados con el fin de controlar el porcentaje
de falsos positivos que puedan resultar del alto numero de contrastes
realizados simultaneamente.
\end{itemize}





La tabla siguiente (\ref{topTable1}) muestra la cabecera de una de las
tablas resultantes del análisis. Las tablas completas se presentan en
archivos anexos a este informe.

% latex table generated in R 3.2.0 by xtable 1.7-4 package
% Thu Jul  9 16:18:18 2015
\begin{longtable}{rrrrrrr}
  \hline
 & logFC & AveExpr & t & P.Value & adj.P.Val & B \\ 
  \hline
20504553 & -217.43 & 221.36 & -4.96 & 0.00 & 0.02 & -4.49 \\ 
  20532575 & -215.43 & 149.55 & -4.81 & 0.00 & 0.02 & -4.49 \\ 
  20503787 & -268.71 & 265.42 & -4.79 & 0.00 & 0.02 & -4.49 \\ 
  20534302 & -213.14 & 153.87 & -4.62 & 0.00 & 0.03 & -4.50 \\ 
  20500737 & 684.57 & 491.84 & 4.55 & 0.00 & 0.03 & -4.50 \\ 
  20501147 & -216.29 & 208.45 & -4.36 & 0.00 & 0.03 & -4.51 \\ 
   \hline
\hline
\caption{Cabecera de la tabla de resultados de la comparación IC.FFvsCL.FF} 
\label{topTable1}
\end{longtable}



Una forma de visualizar los resultados es mediante un \texttt{volcano
  plot} que representa en abscisas los cambios de expresión en escala
logarítmica y en ordenadas el ``menos logaritmo'' del p-valor o
alternativamente el estadístico $B$. Cuanto más abierta está la figura mayor es el efecto biológico y cuanto más arriba estan los puntos (con valor positivo de B, mayor el estadístico).

La figura \ref{fig:volcano11} muestra uno de los ``volcanos'' (comparacion
''OXA1.in.HT29''). El resto se encuentran en el archivo
\texttt{Volcanos.pdf} accesible desde el archivo de resultados. 

Los que aparecen en las figuras números indican los identificadores
``Entrez'' de los genes más diferencialmente expresados.

\begin{knitrout}
\definecolor{shadecolor}{rgb}{0.969, 0.969, 0.969}\color{fgcolor}\begin{figure}
\includegraphics[width=\maxwidth]{images/graficvolcano11-1} \caption[Volcano plot of OXA1 in HT29 cells comparison]{Volcano plot of OXA1 in HT29 cells comparison}\label{fig:volcano11}
\end{figure}


\end{knitrout}

\begin{knitrout}
\definecolor{shadecolor}{rgb}{0.969, 0.969, 0.969}\color{fgcolor}\begin{kframe}


{\ttfamily\noindent\bfseries\color{errorcolor}{\#\# Error in is(fit, "{}MArrayLM"{}): object 'fit.main12' not found}}

{\ttfamily\noindent\bfseries\color{errorcolor}{\#\# Error in int\_abline(a = a, b = b, h = h, v = v, untf = untf, ...): plot.new has not been called yet}}

{\ttfamily\noindent\bfseries\color{errorcolor}{\#\# Error in is(fit, "{}MArrayLM"{}): object 'fit.main12' not found}}

{\ttfamily\noindent\bfseries\color{errorcolor}{\#\# Error in int\_abline(a = a, b = b, h = h, v = v, untf = untf, ...): plot.new has not been called yet}}

{\ttfamily\noindent\bfseries\color{errorcolor}{\#\# Error in is(fit, "{}MArrayLM"{}): object 'fit.main22' not found}}

{\ttfamily\noindent\bfseries\color{errorcolor}{\#\# Error in int\_abline(a = a, b = b, h = h, v = v, untf = untf, ...): plot.new has not been called yet}}

{\ttfamily\noindent\bfseries\color{errorcolor}{\#\# Error in is(fit, "{}MArrayLM"{}): object 'fit.main22' not found}}

{\ttfamily\noindent\bfseries\color{errorcolor}{\#\# Error in int\_abline(a = a, b = b, h = h, v = v, untf = untf, ...): plot.new has not been called yet}}\end{kframe}
\end{knitrout}







\end{document}
